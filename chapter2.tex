\chapter{Periodic Structure of Crystals}
\section{Bravais lattices}
\dfn{Bravais latticees}{There are 14 different lattice types, these are called Bravais lattices. These lattices can be subdivided into 7 different lattice systems, these lattice systems are:
\begin{enumerate}
    \setlength\itemsep{0pt}
    \item   Triclinic
    \item   Monoclinic
    \item   Orthohombric
    \item   Tetragonal
    \item   Cubic
    \item   Triagonal
    \item   Hexagonal
\end{enumerate}
The Cubic structure will mostly be studied during this course.
}

\section{Cubic lattice systems}
Cubic lattice systems come in three flavours, we will define them here. The different systems can be found in figure \ref{fig:cubic_lattice_systems}.
\begin{figure}
    \centering
    \begin{tikzpicture}
		%	Bottom front
		\filldraw [black]	(0, 0) circle (2pt);
		\draw [-*, black]	(0, 0) to (1, 0);

		%	Top front
		\filldraw [black]	(0, 1) circle (2pt);
		\draw [-*, black]	(0, 1) to (1, 1);

		%	Horzontal front
		\draw[--, black]	(0, 0) to (0, 1)
							(0.9, 0) to (0.9, 1);

		%	Top oblique
		\draw[-*, black]	(0, 1) to (0.55, 1.5);
		\draw[-*, black]	(0.9, 1) to (1.45, 1.5);

		%	Side right
		\draw[-*, black]	(1.4, 1.5) to (1.4, 0.4);
		\draw[--, black]	(0.9, 0) to (1.4, 0.5);

		%	Top back
		\draw[--, black]	(0.4, 1.45) to (1.4, 1.45);

		%	Inside
		\draw[dotted, black]	(0.5, 0.5) to (1.4, 0.5)
                                (0.5, 0.5) to (0, 0)
								(0.5, 1.5) to (0.5, 0.5);
		\filldraw[black]	(0.5, 0.5) circle (1pt);

 		\draw [<->, black, thick]	(0, -0.3) to node [below] {a} (0.9, -0.3);
 		\draw [<->, black, thick]	(1.7, 0.5) to node [right] {a} (1.7, 1.5);
 		\draw [<->, black, thick]	(1.1, -0.1) to node [below=1mm] {a} (1.6, 0.4);
	\end{tikzpicture}
%
	\begin{tikzpicture}
		%	Bottom front
		\filldraw [black]	(0, 0) circle (2pt);
		\draw [-*, black]	(0, 0) to (1, 0);

		%	Top front
		\filldraw [black]	(0, 1) circle (2pt);
		\draw [-*, black]	(0, 1) to (1, 1);

		%	Horzontal front
		\draw[--, black]	(0, 0) to (0, 1)
							(0.9, 0) to (0.9, 1);

		%	Top oblique
		\draw[-*, black]	(0, 1) to (0.55, 1.5);
		\draw[-*, black]	(0.9, 1) to (1.45, 1.5);

		%	Side right
		\draw[-*, black]	(1.4, 1.5) to (1.4, 0.4);
		\draw[--, black]	(0.9, 0) to (1.4, 0.5);

		%	Top back
		\draw[--, black]	(0.4, 1.45) to (1.4, 1.45);

		%	Inside
		\draw[dotted, black]	(0.5, 0.5) to (1.4, 0.5)
                                (0.5, 0.5) to (0, 0)
								(0.5, 1.5) to (0.5, 0.5);
		\filldraw[black]	(0.5, 0.5) circle (1pt);

		% Body centering
		\filldraw [gray] (0.7, 0.7) circle (2pt);

 		\draw [<->, black, thick]	(0, -0.3) to node [below] {a} (0.9, -0.3);
 		\draw [<->, black, thick]	(1.7, 0.5) to node [right] {a} (1.7, 1.5);
 		\draw [<->, black, thick]	(1.1, -0.1) to node [below=1mm] {a} (1.6, 0.4);
	\end{tikzpicture}
%
	\begin{tikzpicture}
		%	Face centering
		\filldraw [gray]	(0.95, 0.95) circle (2pt);

		%	Bottom front
		\filldraw [black]	(0, 0) circle (2pt);
		\draw [-*, black]	(0, 0) to (1, 0);

		%	Top front
		\filldraw [black]	(0, 1) circle (2pt);
		\draw [-*, black]	(0, 1) to (1, 1);

		%	Horzontal front
		\draw[--, black]	(0, 0) to (0, 1)
							(0.9, 0) to (0.9, 1);

		%	Top oblique
		\draw[-*, black]	(0, 1) to (0.55, 1.5);
		\draw[-*, black]	(0.9, 1) to (1.45, 1.5);

		%	Side right
		\draw[-*, black]	(1.4, 1.5) to (1.4, 0.4);
		\draw[--, black]	(0.9, 0) to (1.4, 0.5);

		%	Top back
		\draw[--, black]	(0.4, 1.45) to (1.4, 1.45);

		%	Inside
		\draw[dotted, black]	(0.5, 0.5) to (1.4, 0.5)
                                (0.5, 0.5) to (0, 0)
								(0.5, 1.5) to (0.5, 0.5);
		\filldraw[black]	(0.5, 0.5) circle (1pt);

		% Face centering
		\filldraw [gray]	(0.5, 0.5) circle (2pt)
							(0.25, 0.75) circle (2pt)
							(0.7, 1.25) circle (2pt)
 							(0.7, 0.25) circle (2pt)
 							(1.15, 0.7) circle (2pt);

 		\draw [<->, black, thick]	(0, -0.3) to node [below] {a} (0.9, -0.3);
 		\draw [<->, black, thick]	(1.7, 0.5) to node [right] {a} (1.7, 1.5);
 		\draw [<->, black, thick]	(1.1, -0.1) to node [below=1mm] {a} (1.6, 0.4);
	\end{tikzpicture}
    \caption{The three different cubic lattice systems}
    \label{fig:cubic_lattice_systems}
\end{figure}
\dfn{Simple cubic lattice}{A simple cubic lattice is a conventional unit cell and therefore also a \textit{PUC}. This lattice has a straightforward basis, as can be seen in figure \ref{fig:simple_cubic_lattice}.}
The basis chosen is $\{\vec{a}_1, \vec{a}_2, \vec{a}_3\}$. As we can see (figure \ref{fig:simple_cubic_lattice}), the basis isn't body centered. Because the body centered atom is a different one as the other 'side' atoms, the smallest possible unit cell (or \textit{PUC}) is the full cube. Whereas if the middle atom is the same, the basis is chosen in the middle, this is the \textbf{body centered cubic lattice}.
\begin{figure}[h]
    \centering
    \begin{tikzpicture}
		%	Bottom front
		\filldraw [black]	(0, 0) circle (2pt);
		\draw [-*, black]	(0, 0) to (1, 0);

		%	Top front
		\filldraw [black]	(0, 1) circle (2pt);
		\draw [-*, black]	(0, 1) to (1, 1);

		%	Horzontal front
		\draw[--, black]	(0, 0) to (0, 1)
							(0.9, 0) to (0.9, 1);

		%	Top oblique
		\draw[-*, black]	(0, 1) to (0.55, 1.5);
		\draw[-*, black]	(0.9, 1) to (1.45, 1.5);

		%	Side right
		\draw[-*, black]	(1.4, 1.5) to (1.4, 0.4);
		\draw[--, black]	(0.9, 0) to (1.4, 0.5);

		%	Top back
		\draw[--, black]	(0.4, 1.45) to (1.4, 1.45);

		%	Inside
		\draw[dotted, black]	(0.5, 0.5) to (1.4, 0.5)
                                (0.5, 0.5) to (0, 0)
								(0.5, 1.5) to (0.5, 0.5);
		\filldraw[black]	(0.5, 0.5) circle (1pt);

		% Body centering
		\filldraw [cyan] (0.7, 0.7) circle (2pt);

		\draw [->, cyan, thin]	(0.7, 0.7) to node [right=9mm]{different atom} (2.4, 0.7);

 		\draw [<->, black, thick]	(0, -0.3) to node [below] {a} (0.9, -0.3);
 		\draw [<->, black, thick]	(1.7, 0.5) to node [right] {a} (1.7, 1.5);
 		\draw [<->, black, thick]	(1.1, -0.1) to node [below=1mm] {a} (1.6, 0.4);

 		%	Basis
 		\draw [|->, green, very thick]	(0, 0) to (0.9, 0);
 		\draw [|->, red, very thick]	(0, 0) to (0, 1);
 		\draw [|->, blue, very thick]	(0, 0) to (0.48, 0.48);
	\end{tikzpicture}
	\caption{The basis for a simple cubic lattice}
	\label{fig:simple_cubic_lattice}
\end{figure}

\dfn{Body centered cubic lattice}{A Body centered cubic lattice has 1 atom as primitive unit cell, its basis is depectied in figure \ref{fig:bodycubic_lattice_system}. As mentioned before, all atoms are the same and that is why the \textit{PUC} is smaller.}
\begin{figure}[h]
    \centering
	\begin{tikzpicture}
		%	Bottom front
		\filldraw [black]	(0, 0) circle (2pt);
		\draw [-*, black]	(0, 0) to (1, 0);

		%	Top front
		\filldraw [black]	(0, 1) circle (2pt);
		\draw [-*, black]	(0, 1) to (1, 1);

		%	Horzontal front
		\draw[--, black]	(0, 0) to (0, 1)
							(0.9, 0) to (0.9, 1);

		%	Top oblique
		\draw[-*, black]	(0, 1) to (0.55, 1.5);
		\draw[-*, black]	(0.9, 1) to (1.45, 1.5);

		%	Side right
		\draw[-*, black]	(1.4, 1.5) to (1.4, 0.4);
		\draw[--, black]	(0.9, 0) to (1.4, 0.5);

		%	Top back
		\draw[--, black]	(0.4, 1.45) to (1.4, 1.45);

		%	Inside
		\draw[dotted, black]	(0.5, 0.5) to (1.4, 0.5)
                                (0.5, 0.5) to (0, 0)
								(0.5, 1.5) to (0.5, 0.5);
		\filldraw[black]	(0.5, 0.5) circle (1pt);

		% Body centering
		\filldraw [gray] (0.7, 0.7) circle (2pt);

 		\draw [<->, black, thick]	(0, -0.3) to node [below] {a} (0.9, -0.3);
 		\draw [<->, black, thick]	(1.7, 0.5) to node [right] {a} (1.7, 1.5);
 		\draw [<->, black, thick]	(1.1, -0.1) to node [below=1mm] {a} (1.6, 0.4);

 		%	Basis
 		\draw [|->, green, very thick]	(0.7, 0.7) to (0.05, 0.05);
 		\draw [|->, red, very thick]	(0.7, 0.7) to (1.4, 0.5);
 		\draw [|->, blue, very thick]	(0.7, 0.7) to (1, 1);
	\end{tikzpicture}
    \caption{The three different cubic lattice systems}
    \label{fig:bodycubic_lattice_system}
\end{figure}

\dfn{Face centered cubic lattice}{If all atoms are the same and the extra atoms position themselves on the middle of every face, one gets the face centere cubic lattice. This is depicted in figure \ref{fig:facecubic_lattice_system}.}
\begin{figure}[h]
    \centering
	\begin{tikzpicture}
		%	Face centering
		\filldraw [gray]	(0.95, 0.95) circle (2pt);

		%	Bottom front
		\filldraw [black]	(0, 0) circle (2pt);
		\draw [-*, black]	(0, 0) to (1, 0);

		%	Top front
		\filldraw [black]	(0, 1) circle (2pt);
		\draw [-*, black]	(0, 1) to (1, 1);

		%	Horzontal front
		\draw[--, black]	(0, 0) to (0, 1)
							(0.9, 0) to (0.9, 1);

		%	Top oblique
		\draw[-*, black]	(0, 1) to (0.55, 1.5);
		\draw[-*, black]	(0.9, 1) to (1.45, 1.5);

		%	Side right
		\draw[-*, black]	(1.4, 1.5) to (1.4, 0.4);
		\draw[--, black]	(0.9, 0) to (1.4, 0.5);

		%	Top back
		\draw[--, black]	(0.4, 1.45) to (1.4, 1.45);

		%	Inside
		\draw[dotted, black]	(0.5, 0.5) to (1.4, 0.5)
                                (0.5, 0.5) to (0, 0)
								(0.5, 1.5) to (0.5, 0.5);
		\filldraw[black]	(0.5, 0.5) circle (1pt);

		% Face centering
		\filldraw [gray]	(0.5, 0.5) circle (2pt)
							(0.25, 0.75) circle (2pt)
							(0.7, 1.25) circle (2pt)
 							(0.7, 0.25) circle (2pt)
 							(1.15, 0.7) circle (2pt);

 		\draw [<->, black, thick]	(0, -0.3) to node [below] {a} (0.9, -0.3);
 		\draw [<->, black, thick]	(1.7, 0.5) to node [right] {a} (1.7, 1.5);
 		\draw [<->, black, thick]	(1.1, -0.1) to node [below=1mm] {a} (1.6, 0.4);

 		%	Basis
 		\draw [|->, red, very thick]	(0, 0) to (0.25, 0.75);
 		\draw [|->, blue, very thick]	(0, 0) to (0.7, 0.25);
		\draw [|->, green, very thick]	(0, 0) to (0.5, 0.5);
	\end{tikzpicture}
    \caption{The three different cubic lattice systems}
    \label{fig:facecubic_lattice_system}
\end{figure}

\section{C/Si/Ge - lattice systems}
As we know, the lattice systems for C, Si and Ge have a diamond lattice structre. This diamond structure takes the form of a fcc (face centered cubic) lattice.
