\chapter{}
\section{Crystal momentum}
Remember band diagrams derived in previous section \todo{add section}. To introduce the crystal momentum, let's first look at
\begin{align}
	&\hat{H} = \frac{\hat{p}}{2m} = -\frac{\hbar^2}{2m}\nabla^2 with \hat{\vec{p}} = -n\hbar\vec{\nabla}\\
	&\hat{H}\psi_{\vec{k}}(\vec{r}) = E \psi_{\vec{k}}(\vec{r})\\
	&[\hat{H}, \hat{\vec{p}}] = 0\\
	&using de brogle \vec{p} = \hbar\vec{k}
	&\psi = 1/V....
	&\hat{\vec{p}}\psi_{\vec{k}}(\vec{r}) = \hbar\vec{k}\psi_{\vec{k}}(\vec{r})
\end{align}
Bloch electrons: $\hat{H} = p^2/2m + V(vecr)$, with $\hat{\vec{p}} = -i\hbar\vec{\nabla}$. 
Then now it is clear that these operators $[\hat{H}, \hat{\vec{p}}] \neq 0$
Then we can derive the following:
\begin{align}
	Hhat\psi_{n, veck}(vecr) &= E_n(veck)psi..\\
	\psi_{n, \vec{k}}(\vec{r}) &= u_{n, \vec{k}}(\vec{r})e^{i\vec{k}\cdot\vec{r}}\\
	\hat{\vec{p}}\psi_{n, \vec{k}}(\vec{r}) &= -i\hbar\vec{\nabla}\left[u_{n, \vec{k}}(\vec{r})e^{i\vec{k}\cdot\vec{r}}\right] \text{using above relation}\\
	Work it somewhat further out.
\end{align}
How do we interpret the $\vec{k}$? Let's define crystal momentum as follows:
\begin{equation}
	\vec{P} = \hbar\vec{k}
\end{equation}
\nt{Don't confuse $\vec{P}$ with $\vec{p}$. For a bloch electron we cannot say that $\vec{p} = \hbar\vec{k}$.}

Because real space is infinite, we need some boundary condiditions to confine our space.
\section{Boundary condition}
There are two possiblities to impose boundary conditions:
\begin{itemize}
	\item Dirichlet boundary conditions
	\item Born-von Kerman (periodic) boundary conditions
\end{itemize}
We will use periodic boundary conditions to show that ... . \todo{fill}
\ex{1D boundary condition}{
	For the function $\psi(x) = \psi(x + L)$, with $L$ the crystal length. 
	(1)
	We can use some properties of the bloch functions, now we can deduce the following:
	\begin{align}
		\psi(x + L) &= e^{ikL}psi(x) = \psi(x)\\
		&\Rightarrow kNa = 2\pi n \\
		&\Rightarrow k = \frac{2\pi}{L}n
	\end{align}
	We see that the continious $k$ value has become discrete, therefore we get the following figure. As we might expect, for a bigger lattice, we get a more continious spectrum (because $\Delta k$ becomes smaller).
	(2)
	How many lattice points do we have in our first BZ?
	\begin{equation}
		\frac{\frac{2\pi}{a}}{\Delta k} = \frac{\frac{2\pi}{a}}{\frac{2\pi}{Na}} = N = \text{number of unit cells that build the crystal}
	\end{equation}
}
Can we come to the same conclusion in 3D?
\ex{3D boundary condition}{
	The full crystal is defined by $L_1$, $L_2$, $L_3$. Thus we can say that:
	\begin{align}
		\vec{L_1} &= N_1\vec{a}_1\\
		\vec{L_2} &= N_2\vec{a}_2\\
		\vec{L_3} &= N_3\vec{a}_3
	\end{align}
	We can do the same as we did in the 1D example:
	\begin{align}
		\psi(\vec{r} + \vec{L}_j) &= \psi(\vec{r})\\
		&= \psi(\vec{r} + N_j\vec{a}_j)\\
		\text{Using Bloch}\qquad &\Rightarrow \psi(\vec{r}) = e^{i\vec{k}\cdotN_j\vec{a}_j}\psi{\vec{r}}
		&\Rightarrow \vec{k} = 2\pi n \cdot (N_j \vec{a}_j)^{-1} \text{with} n \in \bbZ\\
		&\Rightarrow \vec{k} = \frac{n_1}{N_1}\vec{b}_1 + ...
	\end{align}

	$\vec{b}_i$ are primitive reciprocal lattice vectors.

	Now we look at the elemental volume of the unit cell:
	\begin{align}
		\Delta \vec{k} = \Delta k^3 &= \frac{\vec{b}_1}{N_1}\cdot\left(\frac{\vec{b}_2}{N_2} \cross \frac{\vec{b}_3}{N_3}\right)\\
		&= \frac{1}{N}\vec{b}_1\cdot\left(\vec{b}_2 \cross \vec{b}_3\right)
	\end{align}
	Where N is the number of unit cells that build the crystal.
}
We do get the same result. This has some consequences.
\begin{enumerate}
	\item Even number of electrons in a full shell per unit cell. We get $2N =$ amount of electrons. This gives a semiconductor.
	\item Odd number of electrons in apartially filled shell per unit cell. Because there are still states free in the valence band, gives this a metal.
\end{enumerate}
\nt{This does not mean that a $2N$ amount of electrons gives a semiconductor. Band overlap can still give a metal!}
We now link this concept to real atoms.
\subsection{Atmic picture}
Say, we have $N$ atoms.
For each atom we have the following states: (3)
Now suppose we have them far apart and bring them closer. (4)
This just shows how the bands are for the atoms, nothig can be said about momentum, \dots. What we notice from the picture, too is:
\begin{itemize}
	\item 2s orbital starts to split sooner, because 2s oribitals overlap more as they are furhter away from the nucleus.
	\item We see a smaller splitting at 1s.
\end{itemize}

We will now look at some other atoms.
\ex{Lithium (Li)}{
	For 1 atom we have: (5)
	For $Li_2$ we get: (6)
	For a crystal $Li_N$ we get: (7)
}
\ex{Silicium (Si)}{
	Si has following configuration: $1s^2\,2s^2\,2p^6\,3s^2\,3p^2$
	(8)
	For a Si crystal we have $N$ unit cells. Si cyrstals have a diamond structure with 2 atoms per unit cell. The according band diagram: (9)
	For each unit cell we get: (10)
}
We can represent these band configurations in k-space, too, but these can get complicated. But there is a solution for that. (11)
We will travel along the red line. This gives a plot that has enough information about the crystal. Thus using the high symmetry points is adequate.