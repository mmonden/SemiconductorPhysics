\section{X-ray diffraction}
One way where the 'Von Laue-Bragg' condition can be observed is in x-ray diffraction. The 'Von Laue-Bragg' condition states that an inciding electron is or scattered or stays unscattered when colliding with a crystal. This scattering is elastic thus the amplitude of the wave function does not change. This is illustrated in figure \ref{fig:vonlaue}.
\begin{figure}[b]
    \centering
    \begin{tikzpicture}
        \draw[->, black]	node[left] {$e^-$}(0, 0) to node[above]{incident} node[below]{$\vec{k}$} (2, 0);

        \draw[black] plot [smooth cycle] coordinates {(2.6, 0) (3.2, 0.6) (2.9, 0.7) (3.5, 1) (4.2, 0.9) (4.5, -0.3) (3.9, -0.2) (3.5, -0.7)} node [above=5mm, size=1pt]{CRYSTAL};

        \draw[->, black]	(5, 0.2) to node[above]{$\vec{k}'$}(6, 0.8) node[right=1mm]{$e^-$ (scattered)};
		\draw[->, black]	(5, 0) to node[below]{$\vec{k}$}(6, 0) node[right=1mm]{$e^-$ (unscattered)};
    \end{tikzpicture}
    \caption{X-ray diffraction on a lattice}
    \label{fig:vonlaue}
\end{figure}
Becuase the scattering is elastic we can state the following, define the wave as $e^{i\vec{k}\cdot\vec{r}}$:
\begin{align}
	E(\vec{k}) &= \frac{\hbar^2k^2}{2m}\\
	&= E(\vec{k}')\\
	&= \frac{\hbar^2k'^2}{2m}\\
	&\Rightarrow \abs{\vec{k}} = \abs{\vec{k}'}
\end{align}
One way of figuring out if an electron is scattered is to use the Fermi golden rule.
\section{Fermi golden rule}
abc

\subsection{Von Laue-Bragg condition} \label{sec:bragg}
abc

\section{Von Laue-Bragg condition}
abc

\section{The Brillouin Zone (BZ)} \label{sec:Brillouin}
abc
